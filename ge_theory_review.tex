\documentclass[12pt]{report}

%%METADATA
\title{Notes on General Equilibrium Theory}
\author{
Junbiao Chen\thanks{E-mail: jc14076@nyu.edu.}
}
\date{\today}


%%PACKAGES
\usepackage{mdframed} % For boxed environments
\usepackage{graphicx}
\usepackage{grffile}
\usepackage{tabularx}
\usepackage{setspace}
\usepackage{amsmath,amsthm,amssymb}
\usepackage[hyphens]{url}
\usepackage{natbib}
\usepackage[font=normalsize,labelfont=bf]{caption}
\usepackage[margin=1in]{geometry}
\usepackage{hyperref}
\hypersetup{colorlinks=true,urlcolor=blue,citecolor=blue}
\usepackage{stmaryrd}  %Package with \boxast command
\usepackage{enumerate}% http://ctan.org/pkg/enumerate %Supports lowercase Roman-letter enumeration
\usepackage{verbatim} %Package with \begin{comment} environment
%\usepackage{enumitem}
\usepackage{physics}
\usepackage{tikz}
\usepackage{listings}
\usepackage{upquote}
\usepackage{booktabs} %Package with \toprule and \bottomrule
\usepackage{etoc}     %Package with \localtableofcontents
\usepackage{placeins}    %Package that prevent repositioning the tables
\usepackage{multicol}
\usepackage{bm}
\usepackage{subfig}
\usepackage{csquotes}
\usepackage{tcolorbox}
\tcbset{
    colback=red!10, % Background color of the box
    colframe=black,  % Border color of the box
    boxrule=0.5pt,   % Thickness of the border
    arc=2mm,         % Rounded corners
    left=1mm,        % Left padding
    right=1mm,       % Right padding
    top=1mm,         % Top padding
    bottom=1mm,      % Bottom padding
}
\definecolor{dkgreen}{rgb}{0,0.6,0}
\definecolor{gray}{rgb}{0.5,0.5,0.5}
\definecolor{mauve}{rgb}{0.58,0,0.82}

\lstset{language=bash,
  frame=tb,
  aboveskip=3mm,
  belowskip=3mm,
  showstringspaces=false,
  columns=flexible,
  basicstyle={\small\ttfamily},
  numbers=none,
  numberstyle=\tiny\color{gray},
  keywordstyle=\color{blue},
  commentstyle=\color{dkgreen},
  stringstyle=\color{mauve},
  breaklines=true,
  breakatwhitespace=false,
  tabsize=3
}

\lstset{language=C,
  aboveskip=3mm,
  belowskip=3mm,
  showstringspaces=false,
  columns=flexible,
  basicstyle={\small\ttfamily},
  numbers=none,
  numberstyle=\tiny\color{gray},
  keywordstyle=\color{blue},
  commentstyle=\color{dkgreen},
  stringstyle=\color{mauve},
  breaklines=true,
  breakatwhitespace=false,
  tabsize=4
}

\definecolor{lightblue}{rgb}{0.68, 0.85, 0.9} %

%CUSTOM DEFINITIONS
\theoremstyle{definition}
\newtheorem{definition}{Definition}[section]
\newtheorem*{remark}{Remark}
\setcounter{secnumdepth}{3}
\usepackage{tikz}
\usetikzlibrary{arrows.meta}
\usetikzlibrary{automata, positioning, arrows, calc}

\tikzset{
	->,  % makes the edges directed
	>=stealth, % makes the arrow heads bold
	shorten >=2pt, shorten <=2pt, % shorten the arrow
	node distance=3cm, % specifies the minimum distance between two nodes. Change if n
	every state/.style={draw=blue!55,very thick,fill=blue!20}, % sets the properties for each ’state’ n
	initial text=$ $, % sets the text that appears on the start arrow
 }

%% PROPOSITION
% Define the Proposition environment
\newmdenv[
  innerleftmargin=10pt, 
  innerrightmargin=10pt,
  innertopmargin=10pt,
  innerbottommargin=10pt,
  linecolor=black, 
  linewidth=1pt,
  backgroundcolor=white, 
  roundcorner=5pt
]{propositionbox}

\newtheoremstyle{boldtitle} % Define a new theorem style
  {10pt} % Space above
  {10pt} % Space below
  {\itshape} % Body font
  {} % Indent amount
  {\bfseries} % Theorem head font
  {.} % Punctuation after theorem head
  { } % Space after theorem head
  {} % Theorem head spec

\theoremstyle{boldtitle} % Use the custom style
\newtheorem{proposition}{Proposition} % Define the proposition environment

% Redefine the proposition environment to use the box
\newenvironment{boxedproposition}[1][]
{\begin{propositionbox}\begin{proposition}[#1]}
{\end{proposition}\end{propositionbox}}

%%FORMATTING
\usepackage[bottom]{footmisc}
\onehalfspacing
\numberwithin{equation}{section}
\numberwithin{figure}{section}
\numberwithin{table}{section}
\bibliographystyle{../bib/aeanobold-oxford}

\newtheorem{theorem}{Theorem}
\newcommand{\inner}[2]{\langle #1, #2 \rangle}

%main text

\begin{document}
\maketitle



\tableofcontents
\chapter{Math Preliminaries}

\section{Differential Topology}
\begin{definition}[Transversality]
    Let $f: \mathbb{R}^M \rightarrow \mathbb{R}^N$, with $M > N$.
    We say $f$ is \textbf{transversal} at $0$ (denoted as $f \pitchfork 0$), if 
    \[ \forall x, \text{ such that } f(x) = 0\], we have the Jacobian matrix
    \[
    Df = \begin{bmatrix}
    \frac{\partial f_1}{\partial x_1} & \frac{\partial f_1}{\partial x_2} & \cdots & \frac{\partial f_1}{\partial x_m} \\
    \frac{\partial f_2}{\partial x_1} & \frac{\partial f_2}{\partial x_2} & \cdots & \frac{\partial f_2}{\partial x_m} \\
    \vdots & \vdots & \ddots & \vdots \\
    \frac{\partial f_n}{\partial x_1} & \frac{\partial f_n}{\partial x_2} & \cdots & \frac{\partial f_n}{\partial x_m}
    \end{bmatrix} 
    \] is full rank, namely, rank($Df$) = $N$.
\end{definition}

\noindent \textbf{Example}
Consider $M = 2$, and $N = 1$.
Define
\[
f(x, y) = x^2 + y^2 - 1
\]
The zero set is the unit circle. $x^2 + y^2 = 1$.

Since $n=1$, the Jacobian is a single row vector (the gradient):
\[
Df(x,y) = 
\begin{bmatrix}
    \frac{\partial f}{\partial x} & \frac{\partial f}{\partial y}
\end{bmatrix}
= 
\begin{bmatrix}
    2x & 2y
\end{bmatrix}
\]
The rank of a $1 \times 2$ matrix is 1 (full rank) unless all entries are zero.
\[
Df = [0, 0] \text{ iff } x = 0 \text{ and } y = 0.
\]
However, the origin is \textbf{not} in the zero set. 
Therefore, for every point actually on the circle ($f=0$), the gradient is non-zero.
Hence, we have
\[
f \pitchfork 0
\]

\noindent \textbf{Example 2: Non-transveral}
\[ g(x, y) = y^2 - x^3 \]
The set of points where $y^2 = x^3$. 
This describes a semicubical parabola which has a sharp ``cusp'' (singular point) at the origin.

The Jacobian is given by
\[
Dg(x, y) = 
    \begin{bmatrix}
        \frac{\partial f}{\partial x} & \frac{\partial f}{\partial y}
    \end{bmatrix}
    = 
    \begin{bmatrix}
        -3x^2 & 2y
    \end{bmatrix}
\]
It follows that $Df(0,0)$ at the point $(0,0)$:
Therefore, its rank is \textbf{0} and thus $g$ is not transversal at $0$.
\begin{figure}[htbp]
  \centering
  \includegraphics[width=0.3\textwidth]{entries/images/non_transveral_function_1.png}
  \includegraphics[width=0.3\textwidth]{entries/images/non_transveral_function_2.png}
  \includegraphics[width=0.3\textwidth]{entries/images/non_transveral_function_3.png}
  \caption{A function that is not transversal at the origin}
  \begin{minipage}{0.95\textwidth}{\footnotesize \textsc{Notes}:
   This function depicts $g(x,y) = y^2 - x^3$, which is not transversal at the origin due to the cusp there.
  \par}\end{minipage}
\end{figure}


\begin{proposition}[Transversality Theorem]
    Let $f: \mathbb{R}^M \rightarrow \mathbb{R}^N$ be $C^\infty$, with $M > N$.
    And decompose $x = [x1, x2]$, where $x1 \in \mathbb{R}^{M - N}$, and $x2 \in \mathbb{R}^N$.
    For almost every $x1$, if $f(x1, x2) = 0$, we have 
    \[
    \text{rank}(D_{x2}f) = 
    \begin{bmatrix}
    \frac{\partial f_1}{\partial x_{2,1}} & \frac{\partial f_1}{\partial x_{2,2}} & \cdots & \frac{\partial f_1}{\partial x_{2,N}} \\
    \frac{\partial f_2}{\partial x_{2,1}} & \frac{\partial f_2}{\partial x_{2,2}} & \cdots & \frac{\partial f_2}{\partial x_{2,N}} \\
    \vdots & \vdots & \ddots & \vdots \\
    \frac{\partial  f_N}{\partial x_{2,1}} & \frac{\partial f_N}{\partial x_{2,2}} & \cdots & \frac{\partial f_N}{\partial x_{2,N}} 
    \end{bmatrix}
    = N.
    \]  
\end{proposition}

\paragraph{\textbf{Remarks:}} We will use transverality theorem to prove a bunch of important results in GE theory, 
such as the genericity of regular economies.
\chapter{Abstract Exchange Economies}
\section{Welfare Analysis}
\begin{proposition}[(First Welfare Theorem)]
    In an abstract exchange economy characterized by $w \in \mathbb{R}^{S+1}_{++}$, 
    any competitive equilibrium allocation $x \in \mathbb{R}^{S+1}_{+}$
    is Pareto efficient.
\end{proposition}

\noindent \textbf{Proof:} We prove by contradiction. 
Suppose there exists $y$ that Pareto dominates $x$, then 
$u(y^i_0) + \sum_s \text{Prob}_s u(y^i_s) \geq_P (x^i_0) + \sum_s \text{Prob}_s u(x^i_s) \forall i$,
but for some $j$ 
$u(y^j_0) + \sum_s \text{Prob}_s u(y^j_s) >_P (x^j_0) + \sum_s \text{Prob}_s u(x^j_s)$.
Then by the Walras Law and the Law of Revealed Prferences, we have 
\[
p_0 (y^i_0 - w^j_0) + \sum_s p_s (y^i_s - w^j_s) \geq p_0 (x^i_0 - w^i_0) + \sum_s p_s (x^i_s - w^i_s) = 0, \quad \text{ and }
\]
\[
p_0 (y^j_0 - w^j_0) + \sum_s p_s (y^j_s - w^j_s) > p_0 (x^j_0 - w^j_0) + \sum_s p_s (x^j_s - w^j_s) = 0
\]
As a result, we have 
\[
p_0 \sum_i (y^i_0 - w^i_0) + \sum_s p_s \sum_i (y^i_s - w^i_s) > 0
\]
Since $p_0, p_s$ are strictly positive, we have 
there exists an $s$ such that
\[
\sum_i (y^i_s - w^i_s) > 0,
\]
which implies that $y$ is not feasible. \(\blacksquare\)
\section{Consumption Externalities and Pigouvian Taxes}
In the presence of consumption externalities, an individual's 
utility depends on both her and others' consumptions.
As a result, the centralized allocation (i.e., Negishi's solution) deviates from the decentralized one.
To induce the decentralized allocation to be Pareto Efficient, one can introduce the so-called 
Pigouvian taxes, so that the taxes make individuals behave \textit{as if} they care about others.
The following derives the tax formula.

For simplicity, we assume that the goods 1 does not have consumption exteranlity.
First, the Negishi problem is give by%
\footnote{
    We use superscripts to denote individuals, and subscripts to index goods.
}
\begin{align*}
    \max_{x \in \mathbb{R}^{LI}} \quad & \sum_i \alpha^i U^i(x^i_1, \{x^i_l \}, \{x^{-i}_l \}) \\ 
    \text{subject to:} & \quad \sum_i x^i_l \leq w_l, \quad \forall l
\end{align*}
Note that there are $L$ constraints, so we introduce $L$ Lagrangean Multipliers in the equation below
\[
\mathcal{L} = \sum_i \alpha^i U^i(x^i_1, \{x^i_l \}, \{x^{-i}_l \}) + \sum_l \lambda_l \left(w_l - \sum_i x^i_l  \right)
\]
The first-order conditions are given by 
\begin{align*}
    [x^i_1] & \qquad \alpha^i \frac{\partial U^i}{\partial x^i_1} = \lambda_1 \\
    [x^i_l] & \qquad \sum_j \alpha^j \frac{\partial U^j}{\partial x^i_l} = \lambda_l \quad \text{ for } l \neq 1
\end{align*}


The decentralized consumer's problem is given by 
\begin{align*}
    \max_{x^i \in \mathbb{R}^{I}} \quad &   U^i(x^i_1, \{x^i_l \}, \{x^{-i}_l \}) \\ 
    \text{subject to:} & \quad \sum_l p_l (1 + \tau^i_l) x^i \leq \sum_l p_l w_l^i, \quad \forall l
\end{align*}
The Lagrangean is given by 
\[
\mathcal{L}^i =  U^i(x^i_1, \{x^i_l \}, \{x^{-i}_l \}) +  \mu^i \sum_l \left[p_l w_l^i - p_l (1 + \tau^i_l) x^i \right]
\]
The first-order conditions are given by 
\begin{align*}
    [x^i_1] & \qquad \frac{\partial U^i}{\partial x^i_1} = \mu^i p_1 = \mu^i \qquad \text{ normalization using } \quad p_1 = 1 \\
    [x^i_l] & \qquad \frac{\partial U^i}{\partial x^i_l} = \mu^i p_l (1 + \tau^i_l) \quad \text{ for } l \neq 1
\end{align*}
Combining the these two conditions, we have 
\[
\frac{\partial U^i}{\partial x^i_l} = \frac{\partial U^i}{\partial x^i_1}  p_l (1 + \tau^i_l)
\]

Note that we can decompose the second FOC of the Negishi problem as 
\[
\frac{\partial U^i}{\partial x^i_l} + \sum_{j \neq i} \frac{\alpha^j}{\alpha^i} \frac{\partial U^j}{\partial x^i_l} = \frac{\lambda_l}{\alpha^i} \quad \text{ for } l \neq 1
\]
Therefore, we can equate the equation above with the decentralized condition and compute the tax formula: 
\[
\underbrace{
\frac{\lambda_l}{\alpha_i} - \sum_{j \neq i} \frac{\alpha^j}{\alpha^i} \frac{\partial U^j}{\partial x^i_l}
}_{\text{Negishi's}} = 
\underbrace{
    \frac{\partial U^i}{\partial x^i_1}  p_l (1 + \tau^i_l)
}_{\text{Decentralized consumer's}}
\]
It follows that 
\begin{align*}
    \tau^i_l = \frac{\lambda_l}{p_l \alpha_i \frac{\partial U^i}{\partial x^i_1}} - 1 
        - \sum_{j \neq i}  \frac{\alpha^j}{p_l \alpha^i}  \frac{\frac{\partial U^j}{\partial x^i_l}}{\frac{\partial U^i}{\partial x^i_1}}
\end{align*}
Now assuming $\alpha^i = \frac{1}{\mu^i} = \frac{1}{\frac{\partial U^i}{\partial x^i_1}}$, and $p_l = \lambda_l$, 
the we have 
\begin{tcolorbox}
    \[
    \tau^i_l = -  \frac{1}{p_l } \sum_{j \neq i} \frac{\frac{\partial U^j}{\partial x^i_l}}{\frac{\partial U^j}{\partial x^j_1}}.
    \]
\end{tcolorbox}
\chapter{Financial Market Economies}
\section{No-Arbitrage}
\subsection{No Arbitrage}
There are multiple definitions of the No Arbitrage condition.
Essentially, they all require zero profit on the spot market at time 0, 
and the future financial market no matter what the future state will be.
Formally, the first definition is given by

\begin{tcolorbox}
\textbf{Definition I (No Arbitrage):}
Let $A \in \mathbb{R}^{S \times J}$ be a future return matrix.
Let $z \in \mathbb{R}^{J}$ be a portfolio.
$q^{T} \in \mathbb{R}^{J}$ be a price vector of the portfolio.%
\footnote{
    In this note, only $q$ is a row vector, while other vectors are columns. 
}
For $W = \begin{bmatrix}
  -q \\ A
\end{bmatrix}$,
there is no $z \in \mathbb{R}^{J}$ such that $Wz > 0$.
\\
\begin{itemize}
  \item \textbf{Remark:} Intuitively, $Wz \in \mathbb{R}^{S+1}$ denotes the profit of buying/selling assets on the spot market and all the future financial markets. 
\end{itemize}
\end{tcolorbox}
\noindent \textit{Example:} Consider 
$W = \begin{bmatrix}
-3 & -4  \\ 5 & 0 \\ 2 & 2 \\ 3 & 1
\end{bmatrix}.$
Then take $z = \begin{bmatrix}
1 \\ -1 
\end{bmatrix}$ (namely, buying 1 unit of asset 1 and selling 1 unit of asset 2),
we have 
$Wz = \begin{bmatrix}
1  \\ 5 \\ 0 \\ 2
\end{bmatrix},$
which means the agent can make profit 1 at time 0 (on the spot market), and 
5, 0, 2 at time 1 if $s = 1, 2, 3$, respectively.


The No Arbitrage condition above is (mathematically) equivalent to the following definition
\begin{tcolorbox}
\textbf{Definition II (No Arbitrage):}
\[
\text{span}(W) \cap \mathbb{R}^{S+1}_{+} = \{ \bm{0}\},
\]
where $\text{span}(W) = \{Wz, \forall z \in \mathbb{R}^{J} \}$ denotes the profit of all posible portfolio.
\end{tcolorbox}


With these two definitions in mind, we now introduce the No-Arbitrage theorem
\begin{tcolorbox}
\textbf{Defintion (No-Arbitrage Theorem)}
\[
\text{span}(W) \cap \mathbb{R}^{S+1}_{+} = \{ \bm{0}\} \Leftrightarrow
\exists \hat{\pi} \in \mathbb{R}^{S+1}_{++} \text{ such that } \hat{\pi}  \tau = 0, \forall \tau \in \text{span}(W)
\]
\textbf{Remarks}: 
\begin{itemize}
  \item We require $\hat{\pi} \in \mathbb{R}^{S+1}_{++}$ (strictly positive) due to the need of normalization (see below).
\end{itemize}
\end{tcolorbox}

Intuitively, $\hat{\pi}$ serves as a machinery that summarizes the agent's profit at time 0 and time 1, 
taking probabilities and discount factors into considerations:
\[
\hat{\pi}_0 [Wz]_{0} + \sum_s \hat{\pi}_s  [Wz]_{s} = 0, \quad \forall z \in \mathbb{R}^{J}
\]
Since $[Wz]_{0} = - \sum_j q_j z_j $, after the normalization $\pi_s = \frac{\hat{\pi}_s}{\hat{\pi}_0}$, we have 
\[
- \sum_j q_j z_j  +  \sum_s \pi_s \sum_j a_{sj} z_j = 0,
\]
where $\pi_s$ denotes the price of one unit of consumption good delivered in state $s$ at time 1.
In matrix form, we have $(-q + \pi^{T} A) z = 0 \Rightarrow q = \pi^{T} A$.
\begin{tcolorbox}
Defining $\pi_s = m_s \text{Prob}_s$, we have the \textbf{Fundamental Equation of Asset Pricing}:
\begin{equation}
    \label{eqn:fundamental_asset_pricing_eqn}
    q = \mathbb{E}\left(m A \right).
\end{equation}
\end{tcolorbox}
\textbf{Proof of the No-Arbitrage Theorem}



\begin{tcolorbox}
\textbf{Proposition. (The degree of freedom of stochastic discounts)}
Given $A \in \mathbb{R}^{S\times J}$ of rank $J$ and $q \in \mathbb{R}^J$, the set of viable stochastic discounts 
is an $\mathbb{R}^{S-J}$ strictly positive subspace of $\mathbb{R}^J$, namely,
\[
R(q;A)  = \{\pi \in \mathbb{R}^S_{++}: q = \pi^T A \} = \mathbb{R}^{S-J}_{++}
\]
In words, once you find a viable $\pi$, there are only $S-J$ elements in that $\pi$ you can vary.
\end{tcolorbox}



\section{Arrow Theorem}
\chapter{General Equilibirum with Incomplete Markets (GEI)}
In this chapter, we introduce several examples of the so-called ``Pecuniary Externalities'',
where the prices have dual roles in the economy:
\begin{itemize}
    \item They clear the markets;
    \item They also affect the restrictions on agents' choice set. 
    Concretely, the consumption allocation is restricted by:
    \[
    \begin{bmatrix}
        \vdots \\ x^i_s \\ \vdots 
    \end{bmatrix} \in B^i(p_1, \ldots, p_S).
    \]
\end{itemize}

An important proposition that relates to the second role of prices in the GEI is the following:
\begin{tcolorbox}
\begin{proposition}
    Constrained Pareto Efficiency $\Rightarrow B^i \perp p$.
\end{proposition}
Equivalently, 
If $B^i$ depends on prices $p$, the equilibrium allocation is generally Constrained Inefficient.%
\footnote{
    An exception being homothetic preferences.
}
\end{tcolorbox}

The converse is not true for the proposition above if $L > 1$.
But by the Diamond's theorem, we have if $L=1$ (single good economy), then
\begin{tcolorbox}
\begin{proposition}
    $B^i \perp p$ $\Leftrightarrow$ Constrained Pareto Efficiency \quad \text{ when } $L=1$.
\end{proposition}
\end{tcolorbox}

Consider an economy with agents $i \in \mathcal{I}$, two dates $t=0,1$, and states $s \in \mathcal{S}$ at date 1 with probabilities $\pi_s$.
Let $z \in R^K$ be the asset portfolio vector with date-0 price vector $q$. The asset payoff in state $s$ is $a_s \in R^K$.

\textbf{Consumer $i$'s Problem:}
\begin{align*}
    \max_{\{x^i, z^i\}} \quad & u^i(x_0^i) + \sum_{s \in \mathcal{S}} \pi_s u^i(x_s^i) \\
    \text{s.t.} \quad & p_0 (x_0^i - w_0^i) + q \cdot z^i = 0 \\
                      & p_s (x_s^i - w_s^i) = a_s \cdot z^i, \quad \forall s \in \mathcal{S}
\end{align*}

\section{Sources of Inefficiency}

\subsection{1. Bid--Ask Spread (Transaction Costs)}

\textbf{Inefficiency Analysis:}
\begin{itemize}
    \item \textbf{Real Costs:} If spreads represent resource costs, the allocation is Second-Best Efficient.
    \item \textbf{Market Power/Rents:} If spreads represent rents, the outcome is inefficient relative to the frictionless benchmark (impedes risk sharing).
\end{itemize}

\textbf{Model Modification:}
We distinguish between long ($z^L$) and short ($z^S$) positions. Let $q^a$ be the ask price and $q^b$ be the bid price ($q^a \geq q^b$).

\begin{itemize}
    \item \textbf{Positions:} $z = z^L - z^S$ with $z^L, z^S \geq 0$.
    \item \textbf{Date 0 Budget:}
    \[
    p_0(x_0 - w_0) + q^a \cdot z^L - q^b \cdot z^S = 0
    \]
    \item \textbf{State $s$ Budget:}
    \[
    p_s(x_s - w_s) = a_s \cdot (z^L - z^S)
    \]
\end{itemize}

\subsection{2. Exogenous Borrowing Limits}

\textbf{Inefficiency Analysis:}
\begin{itemize}
    \item Relative to First-Best: \textbf{Yes}, it rules out mutually beneficial intertemporal trades.
    \item Relative to Constrained Set: Typically Constrained Pareto Efficient (CPE) if the planner faces the same enforceability constraint.
\end{itemize}

\textbf{Model Modification:}
Impose a cap $B^i \geq 0$ on the value of short positions (or borrowing).

\begin{itemize}
    \item \textbf{Constraint:}
    \[
    q \cdot z \geq -B^i \quad \text{(Net value limit)}
    \]
    \item \textit{Alternatively, with Bid-Ask spreads:}
    \[
    q^b \cdot z^S \leq B^i
    \]
\end{itemize}

\subsection{3. Default (Limited Enforcement)}

\textbf{Inefficiency Analysis:}
\begin{itemize}
    \item \textbf{Constrained Inefficiency:} Generally yes. Default introduces deadweight losses ($\psi$) and pecuniary externalities (via price-dependent recovery rates).
\end{itemize}

\textbf{Model Modification:}
Agents choose collateralized short positions. Let $r_s$ be the actual repayment on short positions in state $s$.

\begin{itemize}
    \item \textbf{Positions:} $z = z^L - z^S$, with $z^L, z^S \geq 0$.
    \item \textbf{Repayment Constraint (Limited Liability):}
    \[
    0 \leq r_s \leq a_s \cdot z^S
    \]
    \item \textbf{Date 0 Budget:}
    \[
    p_0(x_0 - w_0) + q \cdot (z^L - z^S) = 0
    \]
    \item \textbf{State $s$ Budget:}
    \[
    p_s(x_s - w_s) = a_s \cdot z^L - r_s - \underbrace{\psi_s(a_s \cdot z^S - r_s)}_{\text{Optional deadweight cost}}
    \]
    \item \textbf{Pricing:} Lenders are rational, so $q$ reflects the anticipated repayment schedule $\{r_s\}$.
\end{itemize}

\subsection{4. Collateral (Secured Borrowing)}

\textbf{Inefficiency Analysis:}
\begin{itemize}
    \item \textbf{Pecuniary Externalities:} Yes. The value of collateral depends on spot prices ($P_s^k$). When prices fall, constraints tighten, causing fire-sales which lower prices further. Competitive equilibria are generally Constrained Inefficient.
\end{itemize}

\textbf{Model Modification:}
Introduce a durable, collateralizable asset $k \geq 0$ with date-0 price $p_0^k$ and state-$s$ price $P_s^k$. Only a fraction $\mu_s \in (0,1]$ is pledgeable.

\begin{itemize}
    \item \textbf{Date 0 Budget:}
    \[
    p_0(x_0 - w_0) + p_0^k k + q \cdot (z^L - z^S) = 0
    \]
    \item \textbf{Collateral Constraint (per state $s$):}
    \[
    r_s \leq \mu_s P_s^k k
    \]
    \item \textbf{State $s$ Budget:}
    \[
    p_s(x_s - w_s) = a_s \cdot z^L - r_s + P_s^k k
    \]
    \textit{Note: The term $P_s^k k$ represents the resale value of the collateral asset held by the borrower, net of the repayment $r_s$.}
\end{itemize}

\subsection{5. Information Asymmetry}

\textbf{Inefficiency Analysis:}
\begin{itemize}
    \item \textbf{General Inefficiency:} Markets fail to reach the informationally constrained frontier due to adverse selection (pooling) or moral hazard (contract externalities).
\end{itemize}

\textbf{Model Modification (Two Approaches):}

\begin{enumerate}
    \item \textbf{Non-verifiable States (Radner):} Contracts cannot condition on $s$, but only on a public signal $y = y(s)$. Payoffs $a_{y}$ and prices $p_{y}$ are defined over signals, effectively pooling states with the same $y$.
    \item \textbf{Incentive Compatibility (Mechanism Design):} Keep state-contingent goods but impose Truth-Telling constraints. If state $s$ is private info, for any $s, t$ with same observables:
    \[
    u(x_s) - T_s \geq u(x_t) - T_t
    \]
    where $T_s$ is the net financial transfer in state $s$.
\end{enumerate}

\subsection{6. Pecuniary Externality}

\textbf{Analysis:}
Changes in prices are not market failures in complete markets (First Welfare Theorem). However, in incomplete markets, if prices enter constraints (e.g., Collateral constraints $r_s \leq \mu_s P_s^k k$), price changes affect the feasible set of other agents.
\begin{itemize}
    \item \textbf{Result:} Constrained Inefficiency (Greenwald-Stiglitz / Geanakoplos-Polemarchakis).
    \item \textbf{Modification:} This is captured structurally by \textbf{Section 4 (Collateral)} or any borrowing limit tied to endogenous prices.
\end{itemize}

\subsection{7. Precautionary Saving Motive}

\textbf{Analysis:}
\begin{itemize}
    \item \textbf{Status:} Not a distortion per se. It is an optimal response to uncertainty and convexity of marginal utility ($u''' > 0$).
    \item \textbf{Modification:} No modification to the constraints is required.
\end{itemize}
\chapter{Assignment and Search}


The dual assignment and the primal assignment problems yield the same solution.


\noindent \textbf{Proof}
The Primal problem is given by 
\begin{align*}
    \max_{\pi^{ih}} \quad & \sum_i \sum_h \pi^{ih} U^{ih} \\
    \text{s.t.} \quad & \sum_h \pi^{ih} = w^i, \quad \forall i \\
    & \sum_i \pi^{ih} = f^h, \quad \forall h 
\end{align*}

The Dual problem is given by 
\begin{align*}
    \min_{w^{i}, f^h} \quad & \sum_i  u^i w^i + \sum_h v^h f^h \\
    \text{s.t.} \quad & u^i + v^h \geq U^{ih}, \quad \forall i, h
\end{align*}

The Lagragean for the Primal problem is
\[
\mathcal{L}^P = \sum_i \sum_h  \pi^{ih} U^{ih} + \sum_h v^h \left( f^h - \sum_i \pi^{ih} \right) + \sum_i u^i \left(w^i - \sum_h \pi^{ih} \right)
\]

The Lagragean for the Dual problem is
\[
\mathcal{L}^D = \sum_i  u^i w^i + \sum_h v^h f^h + \sum_i \sum_h \pi^{ih} \left( U^{ih} -  u^i  - v^h \right)
\]

Obivously, when choosing proper Lagragean Multipliers,
namely, using $\{v^h\}, \{ u^i \}$ as the LM for the Primal problem and 
$\{\pi^{ih} \}$ for the Dual problem, 
these two Lagragean coincide with each other and thus yield the same solutions. \(\blacksquare\)


%% =========== %% ========= %%
\bibliography{../bib/notes.bib}

\end{document}