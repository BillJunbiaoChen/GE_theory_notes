\chapter{General Equilibirum with Incomplete Markets (GEI)}
In this chapter, we introduce several examples of the so-called ``Pecuniary Externalities'',
where the prices have dual roles in the economy:
\begin{itemize}
    \item They clear the markets;
    \item They also affect the restrictions on agents' choice set. 
    Concretely, the consumption allocation is restricted by:
    \[
    \begin{bmatrix}
        \vdots \\ x^i_s \\ \vdots 
    \end{bmatrix} \in B^i(p_1, \ldots, p_S).
    \]
\end{itemize}

An important proposition that relates to the second role of prices in the GEI is the following:
\begin{tcolorbox}
\begin{proposition}
    Constrained Pareto Efficiency $\Rightarrow B^i \perp p$.
\end{proposition}
Equivalently, 
If $B^i$ depends on prices $p$, the equilibrium allocation is generally Constrained Inefficient.%
\footnote{
    An exception being homothetic preferences.
}
\end{tcolorbox}

The converse is not true for the proposition above if $L > 1$.
But by the Diamond's theorem, we have if $L=1$ (single good economy), then
\begin{tcolorbox}
\begin{proposition}
    $B^i \perp p$ $\Leftrightarrow$ Constrained Pareto Efficiency \quad \text{ when } $L=1$.
\end{proposition}
\end{tcolorbox}

Consider an economy with agents $i \in \mathcal{I}$, two dates $t=0,1$, and states $s \in \mathcal{S}$ at date 1 with probabilities $\pi_s$.
Let $z \in R^K$ be the asset portfolio vector with date-0 price vector $q$. The asset payoff in state $s$ is $a_s \in R^K$.

\textbf{Consumer $i$'s Problem:}
\begin{align*}
    \max_{\{x^i, z^i\}} \quad & u^i(x_0^i) + \sum_{s \in \mathcal{S}} \pi_s u^i(x_s^i) \\
    \text{s.t.} \quad & p_0 (x_0^i - w_0^i) + q \cdot z^i = 0 \\
                      & p_s (x_s^i - w_s^i) = a_s \cdot z^i, \quad \forall s \in \mathcal{S}
\end{align*}

\section{Sources of Inefficiency}

\subsection{1. Bid--Ask Spread (Transaction Costs)}

\textbf{Inefficiency Analysis:}
\begin{itemize}
    \item \textbf{Real Costs:} If spreads represent resource costs, the allocation is Second-Best Efficient.
    \item \textbf{Market Power/Rents:} If spreads represent rents, the outcome is inefficient relative to the frictionless benchmark (impedes risk sharing).
\end{itemize}

\textbf{Model Modification:}
We distinguish between long ($z^L$) and short ($z^S$) positions. Let $q^a$ be the ask price and $q^b$ be the bid price ($q^a \geq q^b$).

\begin{itemize}
    \item \textbf{Positions:} $z = z^L - z^S$ with $z^L, z^S \geq 0$.
    \item \textbf{Date 0 Budget:}
    \[
    p_0(x_0 - w_0) + q^a \cdot z^L - q^b \cdot z^S = 0
    \]
    \item \textbf{State $s$ Budget:}
    \[
    p_s(x_s - w_s) = a_s \cdot (z^L - z^S)
    \]
\end{itemize}

\subsection{2. Exogenous Borrowing Limits}

\textbf{Inefficiency Analysis:}
\begin{itemize}
    \item Relative to First-Best: \textbf{Yes}, it rules out mutually beneficial intertemporal trades.
    \item Relative to Constrained Set: Typically Constrained Pareto Efficient (CPE) if the planner faces the same enforceability constraint.
\end{itemize}

\textbf{Model Modification:}
Impose a cap $B^i \geq 0$ on the value of short positions (or borrowing).

\begin{itemize}
    \item \textbf{Constraint:}
    \[
    q \cdot z \geq -B^i \quad \text{(Net value limit)}
    \]
    \item \textit{Alternatively, with Bid-Ask spreads:}
    \[
    q^b \cdot z^S \leq B^i
    \]
\end{itemize}

\subsection{3. Default (Limited Enforcement)}

\textbf{Inefficiency Analysis:}
\begin{itemize}
    \item \textbf{Constrained Inefficiency:} Generally yes. Default introduces deadweight losses ($\psi$) and pecuniary externalities (via price-dependent recovery rates).
\end{itemize}

\textbf{Model Modification:}
Agents choose collateralized short positions. Let $r_s$ be the actual repayment on short positions in state $s$.

\begin{itemize}
    \item \textbf{Positions:} $z = z^L - z^S$, with $z^L, z^S \geq 0$.
    \item \textbf{Repayment Constraint (Limited Liability):}
    \[
    0 \leq r_s \leq a_s \cdot z^S
    \]
    \item \textbf{Date 0 Budget:}
    \[
    p_0(x_0 - w_0) + q \cdot (z^L - z^S) = 0
    \]
    \item \textbf{State $s$ Budget:}
    \[
    p_s(x_s - w_s) = a_s \cdot z^L - r_s - \underbrace{\psi_s(a_s \cdot z^S - r_s)}_{\text{Optional deadweight cost}}
    \]
    \item \textbf{Pricing:} Lenders are rational, so $q$ reflects the anticipated repayment schedule $\{r_s\}$.
\end{itemize}

\subsection{4. Collateral (Secured Borrowing)}

\textbf{Inefficiency Analysis:}
\begin{itemize}
    \item \textbf{Pecuniary Externalities:} Yes. The value of collateral depends on spot prices ($P_s^k$). When prices fall, constraints tighten, causing fire-sales which lower prices further. Competitive equilibria are generally Constrained Inefficient.
\end{itemize}

\textbf{Model Modification:}
Introduce a durable, collateralizable asset $k \geq 0$ with date-0 price $p_0^k$ and state-$s$ price $P_s^k$. Only a fraction $\mu_s \in (0,1]$ is pledgeable.

\begin{itemize}
    \item \textbf{Date 0 Budget:}
    \[
    p_0(x_0 - w_0) + p_0^k k + q \cdot (z^L - z^S) = 0
    \]
    \item \textbf{Collateral Constraint (per state $s$):}
    \[
    r_s \leq \mu_s P_s^k k
    \]
    \item \textbf{State $s$ Budget:}
    \[
    p_s(x_s - w_s) = a_s \cdot z^L - r_s + P_s^k k
    \]
    \textit{Note: The term $P_s^k k$ represents the resale value of the collateral asset held by the borrower, net of the repayment $r_s$.}
\end{itemize}

\subsection{5. Information Asymmetry}

\textbf{Inefficiency Analysis:}
\begin{itemize}
    \item \textbf{General Inefficiency:} Markets fail to reach the informationally constrained frontier due to adverse selection (pooling) or moral hazard (contract externalities).
\end{itemize}

\textbf{Model Modification (Two Approaches):}

\begin{enumerate}
    \item \textbf{Non-verifiable States (Radner):} Contracts cannot condition on $s$, but only on a public signal $y = y(s)$. Payoffs $a_{y}$ and prices $p_{y}$ are defined over signals, effectively pooling states with the same $y$.
    \item \textbf{Incentive Compatibility (Mechanism Design):} Keep state-contingent goods but impose Truth-Telling constraints. If state $s$ is private info, for any $s, t$ with same observables:
    \[
    u(x_s) - T_s \geq u(x_t) - T_t
    \]
    where $T_s$ is the net financial transfer in state $s$.
\end{enumerate}

\subsection{6. Pecuniary Externality}

\textbf{Analysis:}
Changes in prices are not market failures in complete markets (First Welfare Theorem). However, in incomplete markets, if prices enter constraints (e.g., Collateral constraints $r_s \leq \mu_s P_s^k k$), price changes affect the feasible set of other agents.
\begin{itemize}
    \item \textbf{Result:} Constrained Inefficiency (Greenwald-Stiglitz / Geanakoplos-Polemarchakis).
    \item \textbf{Modification:} This is captured structurally by \textbf{Section 4 (Collateral)} or any borrowing limit tied to endogenous prices.
\end{itemize}

\subsection{7. Precautionary Saving Motive}

\textbf{Analysis:}
\begin{itemize}
    \item \textbf{Status:} Not a distortion per se. It is an optimal response to uncertainty and convexity of marginal utility ($u''' > 0$).
    \item \textbf{Modification:} No modification to the constraints is required.
\end{itemize}