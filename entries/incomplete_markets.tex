\chapter{General Equilibirum with Incomplete Markets (Primer)}


\section{Constrained PE in production economies with incomplete markets ($L = 1$)}
\subsection{Takeaway}
Consider a financial market economy that is already Constrained Pareto efficient,
adding production and equity trades 
\begin{itemize}
    \item does NOT induce inefficiency if firms adopt Minkowski's rational price conjecture;
    \item induces inefficiency under Dr\'erez price conjecture.
\end{itemize}








\section{Pecuniary externalities and constrained inefficiency}
In this section, we introduce several examples of the so-called ``Pecuniary Externalities'',
where the prices have dual roles in the economy:
\begin{itemize}
    \item They clear the markets;
    \item They also affect the restrictions on agents' choice set. 
    Concretely, the consumption allocation is restricted by:
    \[
    \begin{bmatrix}
        \vdots \\ x^i_s \\ \vdots 
    \end{bmatrix} \in B^i(p_1, \ldots, p_S).
    \]
\end{itemize}

An important proposition that relates to the second role of prices in the GEI is the following:
\begin{tcolorbox}
\begin{proposition}
    Constrained Pareto Efficiency $\Rightarrow B^i \perp p$.
\end{proposition}
Equivalently, 
If $B^i$ depends on prices $p$, the equilibrium allocation is generally Constrained Inefficient.%
\footnote{
    An exception is homothetic preferences.
}
\end{tcolorbox}

The converse is not true for the proposition above if $L > 1$.
But by the Diamond's theorem, we have if $L=1$ (single good economy), then
\begin{tcolorbox}
\begin{proposition}
    $B^i \perp p$ $\Leftrightarrow$ Constrained Pareto Efficiency \quad \text{ when } $L=1$.
\end{proposition}
\end{tcolorbox}

Consider an economy with agents $i \in \mathcal{I}$, two dates $t=0,1$, and states $s \in \mathcal{S}$ at date 1 with probabilities $\pi_s$.
Let $z \in R^K$ be the asset portfolio vector with date-0 price vector $q$. The asset payoff in state $s$ is $a_s \in R^K$.

\textbf{Consumer $i$'s Problem:}
\begin{align*}
    \max_{\{x^i, z^i\}} \quad & u^i(x_0^i) + \sum_{s \in \mathcal{S}} \pi_s u^i(x_s^i) \\
    \text{s.t.} \quad & p_0 (x_0^i - w_0^i) + q \cdot z^i = 0 \\
                      & p_s (x_s^i - w_s^i) = a_s \cdot z^i, \quad \forall s \in \mathcal{S}
\end{align*}

\section{Default}
\subsection{Environment}
Consider an economy with $I$ agents, 
$S$ states, a single commodity ($L = 1$), 
and a full set of Arrow securities: $A = I_S$ (where $I_S$ is an S-dimensional identity matrix.)
Agents can default in each state $s \in $, 
after the state is realized. 
If they default they consume only a fraction $\alpha \in (0,1)$ of their endowment. 
The remaining fraction, $1 - \alpha $, is first pooled across all defaulting agents 
and then redistributed pro-rata to their creditors.

\subsection{Equilibirum}
\noindent \textbf{Definition:} A competitive equilibrium with default under complete financial market is defined by 
asset prices $(q \in \mathbb{R}^S_{++})$, consumption choices $x^i \in \mathbb{R}_{++}$, 
and default choices $d^i \in \{0, 1\}$, such that 
\begin{enumerate}
    \item Each consumer solves
    \begin{align*}
        \max_{x^i_0, x^i_s, d^i} \quad & u^i(x^i_0) + \sum_s \text{Prob}_s u^i(x^i_s) \\ 
        \text{subject to: } \quad & x^i_0 - w^i_0 + q z^i = 0 \\ 
        & x^i_s = \begin{cases}
            w^i_s + \underbrace{\frac{z^i_s}{\sum_i \max\{0, z^i_s \}}}_{\text{pro-rata weight}}
            \sum_{j \neq i}  \bigg[d_j (1 - \alpha)w^j_s + (1 - d_j) \underbrace{\max \{0 , - z^j_s\}}_{\text{identifying borrowers}} \bigg] \quad \forall s, \quad \text{ if }  z^i_s > 0 \quad  \text{($i$ is a lender)}\\
            \\
            d_i \alpha w^i_s + (1 - d_i) \left(z^i_s + w^i_s\right)  \quad \forall s, \quad \text{ if } z^i_s < 0 \quad \text{ ($i$ is a borrower)}
        \end{cases}
    \end{align*}
    where I use the state index $s$ to denote the asset as the financial market is complete.
    \item Goods market clears
    \[
    \sum_i x^i = \sum_i w^i 
    \]
    \item Asset market clears
    \[
    \sum_i z^i = 0
    \]
\end{enumerate}