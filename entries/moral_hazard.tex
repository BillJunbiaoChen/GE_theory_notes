\chapter{Moral Hazard}
\section{Prescott-Townsend Competitive Equilibrium}
\noindent \textbf{Definition.} 
A competitive equilibrium in a Moral Hazard Economy is defined by 
an allocation $(x, e) \in \mathbb{R}^S \times E$, 
and prices $q \in \mathbb{R}^{SE}$ such that 
\begin{enumerate}
    \item Each consumer $i$ solves 
    \begin{align*}
        \max_{x^i, e} \quad & \sum_s \text{Prob}_s(e) u(x_s) - v(e) \\ 
        \text{subject to: } \quad & \sum_s q_s(e) (x_s - w_s ) = 0 
    \end{align*}
    \item Market clears 
    \[
    \sum_s \text{Prob}_s(e) (x^i_s - w^i_s) = 0,
    \]
    where due to the Law of Large Number, $\text{Prob}_s(e)$ is the fraction of 
    consumers choose effort $e$ when state $s$ is realized.
    \item Self-imposing incentive constraint:
    \[
    e \in \text{argmax}_\epsilon \sum_s \text{Prob}_s(\epsilon) u(x_s) - v(\epsilon), \quad \text{given} x
    \]
    \item Rational conjecture
    \[
        \text{Prob}_s(e) = q_s(e)
    \]
\end{enumerate}