\chapter{Math Preliminaries}

\section{Differential Topology}
\begin{definition}[Transversality]
    Let $f: \mathbb{R}^M \rightarrow \mathbb{R}^N$, with $M > N$.
    We say $f$ is \textbf{transversal} at $0$ (denoted as $f \pitchfork 0$), if 
    \[ \forall x, \text{ such that } f(x) = 0\], we have the Jacobian matrix
    \[
    Df = \begin{bmatrix}
    \frac{\partial f_1}{\partial x_1} & \frac{\partial f_1}{\partial x_2} & \cdots & \frac{\partial f_1}{\partial x_m} \\
    \frac{\partial f_2}{\partial x_1} & \frac{\partial f_2}{\partial x_2} & \cdots & \frac{\partial f_2}{\partial x_m} \\
    \vdots & \vdots & \ddots & \vdots \\
    \frac{\partial f_n}{\partial x_1} & \frac{\partial f_n}{\partial x_2} & \cdots & \frac{\partial f_n}{\partial x_m}
    \end{bmatrix} 
    \] is full rank, namely, rank($Df$) = $N$.
\end{definition}

\noindent \textbf{Example}
Consider $M = 2$, and $N = 1$.
Define
\[
f(x, y) = x^2 + y^2 - 1
\]
The zero set is the unit circle. $x^2 + y^2 = 1$.

Since $n=1$, the Jacobian is a single row vector (the gradient):
\[
Df(x,y) = 
\begin{bmatrix}
    \frac{\partial f}{\partial x} & \frac{\partial f}{\partial y}
\end{bmatrix}
= 
\begin{bmatrix}
    2x & 2y
\end{bmatrix}
\]
The rank of a $1 \times 2$ matrix is 1 (full rank) unless all entries are zero.
\[
Df = [0, 0] \text{ iff } x = 0 \text{ and } y = 0.
\]
However, the origin is \textbf{not} in the zero set. 
Therefore, for every point actually on the circle ($f=0$), the gradient is non-zero.
Hence, we have
\[
f \pitchfork 0
\]

\noindent \textbf{Example 2: Non-transveral}
\[ g(x, y) = y^2 - x^3 \]
The set of points where $y^2 = x^3$. 
This describes a semicubical parabola which has a sharp ``cusp'' (singular point) at the origin.

The Jacobian is given by
\[
Dg(x, y) = 
    \begin{bmatrix}
        \frac{\partial f}{\partial x} & \frac{\partial f}{\partial y}
    \end{bmatrix}
    = 
    \begin{bmatrix}
        -3x^2 & 2y
    \end{bmatrix}
\]
It follows that $Df(0,0)$ at the point $(0,0)$:
Therefore, its rank is \textbf{0} and thus $g$ is not transversal at $0$.
\begin{figure}[htbp]
  \centering
  \includegraphics[width=0.3\textwidth]{entries/images/non_transveral_function_1.png}
  \includegraphics[width=0.3\textwidth]{entries/images/non_transveral_function_2.png}
  \includegraphics[width=0.3\textwidth]{entries/images/non_transveral_function_3.png}
  \caption{A function that is not transversal at the origin}
  \begin{minipage}{0.95\textwidth}{\footnotesize \textsc{Notes}:
   This function depicts $g(x,y) = y^2 - x^3$, which is not transversal at the origin due to the cusp there.
  \par}\end{minipage}
\end{figure}


\begin{proposition}[Transversality Theorem]
    Let $f: \mathbb{R}^M \rightarrow \mathbb{R}^N$ be $C^\infty$, with $M > N$.
    And decompose $x = [x1, x2]$, where $x1 \in \mathbb{R}^{M - N}$, and $x2 \in \mathbb{R}^N$.
    For almost every $x1$, if $f(x1, x2) = 0$, we have 
    \[
    \text{rank}(D_{x2}f) = 
    \begin{bmatrix}
    \frac{\partial f_1}{\partial x_{2,1}} & \frac{\partial f_1}{\partial x_{2,2}} & \cdots & \frac{\partial f_1}{\partial x_{2,N}} \\
    \frac{\partial f_2}{\partial x_{2,1}} & \frac{\partial f_2}{\partial x_{2,2}} & \cdots & \frac{\partial f_2}{\partial x_{2,N}} \\
    \vdots & \vdots & \ddots & \vdots \\
    \frac{\partial  f_N}{\partial x_{2,1}} & \frac{\partial f_N}{\partial x_{2,2}} & \cdots & \frac{\partial f_N}{\partial x_{2,N}} 
    \end{bmatrix}
    = N.
    \]  
\end{proposition}

\paragraph{\textbf{Remarks:}} We will use transverality theorem to prove a bunch of important results in GE theory, 
such as the genericity of regular economies.