\chapter{Financial Market Economies}
\section{No-Arbitrage}
\subsection{No Arbitrage}
There are multiple definitions of the No Arbitrage condition.
Essentially, they all require zero profit on the spot market at time 0, 
and the future financial market no matter what the future state will be.
Formally, the first definition is given by

\begin{tcolorbox}
\textbf{Definition I (No Arbitrage):}
Let $A \in \mathbb{R}^{S \times J}$ be a future return matrix.
Let $z \in \mathbb{R}^{J}$ be a portfolio.
$q^{T} \in \mathbb{R}^{J}$ be a price vector of the portfolio.%
\footnote{
    In this note, only $q$ is a row vector, while other vectors are columns. 
}
For $W = \begin{bmatrix}
  -q \\ A
\end{bmatrix}$,
there is no $z \in \mathbb{R}^{J}$ such that $Wz > 0$.
\\
\begin{itemize}
  \item \textbf{Remark:} Intuitively, $Wz \in \mathbb{R}^{S+1}$ denotes the profit of buying/selling assets on the spot market and all the future financial markets. 
\end{itemize}
\end{tcolorbox}
\noindent \textit{Example:} Consider 
$W = \begin{bmatrix}
-3 & -4  \\ 5 & 0 \\ 2 & 2 \\ 3 & 1
\end{bmatrix}.$
Then take $z = \begin{bmatrix}
1 \\ -1 
\end{bmatrix}$ (namely, buying 1 unit of asset 1 and selling 1 unit of asset 2),
we have 
$Wz = \begin{bmatrix}
1  \\ 5 \\ 0 \\ 2
\end{bmatrix},$
which means the agent can make profit 1 at time 0 (on the spot market), and 
5, 0, 2 at time 1 if $s = 1, 2, 3$, respectively.


The No Arbitrage condition above is (mathematically) equivalent to the following definition
\begin{tcolorbox}
\textbf{Definition II (No Arbitrage):}
\[
\text{span}(W) \cap \mathbb{R}^{S+1}_{+} = \{ \bm{0}\},
\]
where $\text{span}(W) = \{Wz, \forall z \in \mathbb{R}^{J} \}$ denotes the profit of all posible portfolio.
\end{tcolorbox}


With these two definitions in mind, we now introduce the No-Arbitrage theorem
\begin{tcolorbox}
\textbf{Defintion (No-Arbitrage Theorem)}
\[
\text{span}(W) \cap \mathbb{R}^{S+1}_{+} = \{ \bm{0}\} \Leftrightarrow
\exists \hat{\pi} \in \mathbb{R}^{S+1}_{++} \text{ such that } \hat{\pi}  \tau = 0, \forall \tau \in \text{span}(W)
\]
\textbf{Remarks}: 
\begin{itemize}
  \item We require $\hat{\pi} \in \mathbb{R}^{S+1}_{++}$ (strictly positive) due to the need of normalization (see below).
\end{itemize}
\end{tcolorbox}

Intuitively, $\hat{\pi}$ serves as a machinery that summarizes the agent's profit at time 0 and time 1, 
taking probabilities and discount factors into considerations:
\[
\hat{\pi}_0 [Wz]_{0} + \sum_s \hat{\pi}_s  [Wz]_{s} = 0, \quad \forall z \in \mathbb{R}^{J}
\]
Since $[Wz]_{0} = - \sum_j q_j z_j $, after the normalization $\pi_s = \frac{\hat{\pi}_s}{\hat{\pi}_0}$, we have 
\[
- \sum_j q_j z_j  +  \sum_s \pi_s \sum_j a_{sj} z_j = 0,
\]
where $\pi_s$ denotes the price of one unit of consumption good delivered in state $s$ at time 1.
In matrix form, we have $(-q + \pi^{T} A) z = 0 \Rightarrow q = \pi^{T} A$.
\begin{tcolorbox}
Defining $\pi_s = m_s \text{Prob}_s$, we have the \textbf{Fundamental Equation of Asset Pricing}:
\begin{equation}
    \label{eqn:fundamental_asset_pricing_eqn}
    q = \mathbb{E}\left(m A \right).
\end{equation}
\end{tcolorbox}
\textbf{Proof of the No-Arbitrage Theorem}



\begin{tcolorbox}
\textbf{Proposition. (The degree of freedom of stochastic discounts)}
Given $A \in \mathbb{R}^{S\times J}$ of rank $J$ and $q \in \mathbb{R}^J$, the set of viable stochastic discounts 
is an $\mathbb{R}^{S-J}$ strictly positive subspace of $\mathbb{R}^J$, namely,
\[
R(q;A)  = \{\pi \in \mathbb{R}^S_{++}: q = \pi^T A \} = \mathbb{R}^{S-J}_{++}
\]
In words, once you find a viable $\pi$, there are only $S-J$ elements in that $\pi$ you can vary.
\end{tcolorbox}



\section{Arrow Theorem}