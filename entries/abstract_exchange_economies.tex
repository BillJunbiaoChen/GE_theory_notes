\chapter{Abstract Exchange Economies}
\section{Pareto Efficient Allocations}
\begin{proposition}[Negishi Theorem]
    
\end{proposition}



\section{Welfare Analysis}
\begin{proposition}[First Welfare Theorem]
    In an abstract exchange economy characterized by $w \in \mathbb{R}^{S+1}_{++}$, 
    any competitive equilibrium allocation $x \in \mathbb{R}^{S+1}_{+}$
    is Pareto efficient.
\end{proposition}

\noindent \textbf{Proof:} We prove by contradiction. 
Suppose there exists $y$ that Pareto dominates $x$, then 
$y^i \succeq_P x^i \forall i$,
and  
$y^j \succ_P x^j$ for some $j$.
Then by the Walras Law and the Law of Revealed Prferences, we have 
\[
p_0 (y^i_0 - w^j_0) + \sum_s p_s (y^i_s - w^j_s) \geq p_0 (x^i_0 - w^i_0) + \sum_s p_s (x^i_s - w^i_s) = 0, \quad \text{ and }
\]
\[
p_0 (y^j_0 - w^j_0) + \sum_s p_s (y^j_s - w^j_s) > p_0 (x^j_0 - w^j_0) + \sum_s p_s (x^j_s - w^j_s) = 0
\]
Summing over $i$, we have 
\[
p_0 \sum_i (y^i_0 - w^i_0) + \sum_s p_s \sum_i (y^i_s - w^i_s) > 0
\]
Since $p_0, p_s$ are strictly positive, 
there exists an $s$ such that
\[
\sum_i (y^i_s - w^i_s) > 0,
\]
which implies that $y$ is not feasible. \(\blacksquare\)

\textcolor{blue}{The converse of the First welfare theorem is the Second welfare theorem below.}
\noindent \textbf{Interpretation:}
In words, the Second Welfare Theorem states that 
any Pareto efficient allocation can be decentralized as a competitive equilibrium via 
transfer (equivalently, redistribution of initial endowment, where the transferred endowment 
$\Delta w^i = \frac{T_i}{p}$) --- See MWG Figure 15.b.13 for an illustration.

\begin{proposition}[Second Welfare Theorem]
    For any economy characterized by $w \in \mathbb{R}^{LI}_{++}$, 
    any Pareto efficient allocation $x \in \mathbb{R}^{LI}_{+}$
    can be decentralized as a competitive equilibrium via price, $p \in \mathbb{R}^L_{++}$,
    under redistributed endowments 
    with $\sum_i w_i + t_i = w$ (i.e., $\sum_i t_i = 0$).
\end{proposition}

\noindent \textbf{Proof} 
Remark: The key of the proof is a separating hyperplane argument.
\\
\noindent 
Let $x \in \mathbb{R}^{LI}_{++}$ be a Pareto Efficient allocation.
Choose $w = x$.
Define the strict upper contour set for agent $i$, 
\[
B^i(x_i) := \{y_i \in \mathbb{R}^L_{+}: U(y_i) > U(x_i) \}
\]
Define the aggregate strict upper contour set as 
\[
B(x) := \bigg\{\sum_i y_i \in \mathbb{R}^L_{+}: y_i \in B^i(x_i) \forall i \bigg\}
\]
Since $B^i(x_i)$ is convex, the aggregate upper contour set is also convex. 
In addition, by Supporting Hyperplane theorem, 
there exists a nonzero price vector $p$, such that
\[
\inf_{a \in B(x)} p a \geq p \sum_i x_i
\]
We will use this equation to establish that 
\begin{itemize}
    \item[(1)] $p \in \mathbb{R}^{L}_{++}$, and 
    \item[(2)] under such price, no individual will trade.
\end{itemize}

\noindent \textbf{Part 1.}
Suppose there exists some $l$ such that $p_l < 0$, 
let 
\[
y_k^i = \begin{cases}
    x_k^i + \delta  \quad &\text{ if } k = l \\
    x_k^i \quad & \text{ if } k \neq l \\
\end{cases}
\]
Then, $p y^i < p x^i$ and $y^i \in B^i(x^i)$. 
Summing across agents, we have a contradtion to the supporting hyperplane condtion above.


Now suppose there exists some $l$ such that $p_l = 0$, 
let 
\[
y_k^i = \begin{cases}
    x_k^i + \delta  \quad &\text{ if } k = l \\
    x_k^i - \epsilon \quad & \text{ if } k \neq l \\
\end{cases}
\]
where $\delta$ is sufficiently large and $\epsilon$ is sufficiently small, 
so that the loss in $k \neq l$ can be compensated. 
Then, $p y^i < p x^i$ and $y^i \in B^i(x^i)$. 
Summing across agents, again, we have a contradiction to the supporting hyperplane condtion above.


\noindent \textbf{Part 2.} 
Finally, we prove that under $p$ found by the supporting hyperplane theorem, 
the equilibrium is autarky.
Suppose there is a strictly profitable deviation from $x^i$ for agent $i$,
namely, 
\[
\exists \tilde{x}^i, \quad \text{s.t. } p \tilde{x}^i < p x^i \quad \text{ and } \quad 
\tilde{x}^i \in B^i(x^i)
\]
Denote the payoff from the deviation by $\delta := p x^i - p \tilde{x}^i > 0$.
Pick $\tilde{x}^i + \sum_{j \neq i } y^j \in B(x)$, and apply the SHT, to obtain
\[
p \left(\tilde{x}^i + \sum_{j \neq i } y^j \right) \geq p \sum_i x^i 
\] 
It follows that 
\[
p \left(  \sum_{j \neq i } y^j  - x^j \right) \geq \delta > 0 
\]
However, consider $y^j = x^j + \epsilon$, 
we have 
$\lim p y^j = \lim p x^j$ as $\epsilon \rightarrow 0$,
a contradiction. 
Hence, it cannot exist any profitable deviation for any agent $i$. 
\(\blacksquare\)


\section{Sufficient conditions for the uniqueness of competitive equilibrium}



\section{Sonnenschein-Debreu-Mantel Theorem (Anything-goes)}
The following famous theorem is known as the ``anything-goes'' theorem and Indeterminacy Theorem, 
in the sense that,
one can take ``any'' arbitrary aggregate excess demand function ($z$), 
and construct an economy with rational individuals that matches it.
This proves that individual rationality imposes no restrictions on the aggregate market behavior.
This undertermineness at the macro level outcomes leads to a famous quote by Tom Sargent:
\begin{tcolorbox}
    \quad ``The real problem in economics is not that our models impose too many restrictions on the data, but far too few''.
\end{tcolorbox}
Jesús Fernández-Villaverde says: SMD teaches us that meaningful economic predictions require disciplined structure beyond individual rationality.



\begin{proposition}[Sonnenschein-Debreu-Mantel Theorem]
    Let $\Delta_\epsilon^{L-1}$ be a trimmed simplex, with $\epsilon > 0$. 
    Given any continuous function $z : \Delta_\epsilon^{L-1}$
    satisfying continuity and Walras law,
    then there is an economy with $U^i: \mathbb{R}^L_{+} \rightarrow \mathbb{R}$, 
    $\omega \in  \mathbb{R}^{LI}_{++}$, $I \geq L$, 
    whose aggregate excess demand function, restricted to $\Delta_\epsilon^{L-1}$, equals $z$.
\end{proposition}

