\chapter{Abstract Exchange Economies}
\section{Pareto Efficient Allocations}
\begin{proposition}[Negishi Theorem]
    
\end{proposition}



\section{Welfare Analysis}
\begin{proposition}[First Welfare Theorem]
    In an abstract exchange economy characterized by $w \in \mathbb{R}^{S+1}_{++}$, 
    any competitive equilibrium allocation $x \in \mathbb{R}^{S+1}_{+}$
    is Pareto efficient.
\end{proposition}

\noindent \textbf{Proof:} We prove by contradiction. 
Suppose there exists $y$ that Pareto dominates $x$, then 
$y^i \succeq_P x^i \forall i$,
and  
$y^j \succ_P x^j$ for some $j$.
Then by the Walras Law and the Law of Revealed Prferences, we have 
\[
p_0 (y^i_0 - w^j_0) + \sum_s p_s (y^i_s - w^j_s) \geq p_0 (x^i_0 - w^i_0) + \sum_s p_s (x^i_s - w^i_s) = 0, \quad \text{ and }
\]
\[
p_0 (y^j_0 - w^j_0) + \sum_s p_s (y^j_s - w^j_s) > p_0 (x^j_0 - w^j_0) + \sum_s p_s (x^j_s - w^j_s) = 0
\]
Summing over $i$, we have 
\[
p_0 \sum_i (y^i_0 - w^i_0) + \sum_s p_s \sum_i (y^i_s - w^i_s) > 0
\]
Since $p_0, p_s$ are strictly positive, 
there exists an $s$ such that
\[
\sum_i (y^i_s - w^i_s) > 0,
\]
which implies that $y$ is not feasible. \(\blacksquare\)

\textcolor{blue}{The converse of the First welfare theorem is the Second welfare theorem below.}
\noindent \textbf{Interpretation:}
In words, the Second Welfare Theorem states that 
any Pareto efficient allocation can be decentralized as a competitive equilibrium via 
transfer (AKA redistribution of initial endowment).

\begin{proposition}[Second Welfare Theorem]
    For any economy characterized by $w \in \mathbb{R}^{LI}_{++}$, 
    any Pareto efficient allocation $x \in \mathbb{R}^{LI}_{+}$
    can be decentralized as a competitive equilibrium via price, $p \in \mathbb{R}^L_{++}$,
    under redistributioned endowments 
    with $\sum_i w_i + t_i = w$ (i.e., $\sum_i t_i = 0$).
\end{proposition}

\noindent \textbf{Proof} 
Remark: The key of the proof is a separating hyperplane argument that
relies on the convexity, contintuity, and strict monotonicity of utility functions.
\\
\noindent Let $x_i = w_i + t_i$.
First, define the strict upper contour set for agent $i$, 
\[
B^i(x_i) := \{y_i \in \mathbb{R}^L_{+}: U(y_i) > U(x_i) \}
\]
Then, define the aggregate strict upper contour set as 
\[
B(x) := \bigg\{\sum_i y_i \in \mathbb{R}^L_{+}: y_i \in B^i(x_i) \forall i \bigg\}
\]
Since $B^i(x_i)$ is convex, the aggregate upper contour set is also convex. 
In addition, $(\sum_i x_i) \notin B(x)$ due to the strictness of the upper contour set. 
Then, by the Separting Hyperplane theorem, there exists a nonzero price vector $p$, such that
\[
\inf_{a \in B(x)} p a \geq p \sum_i x_i
\]
We will use this equation to establish that 
(1) prices are strictly positive,%
\footnote{Intuitively, valued goods should not have non-positive prices.}
and 
(2) under such price, no individual will trade away from $x_i$.

\noindent \textbf{Part 1.}
Suppose there exists some $l$ such that $p_l < 0$, 
by the fact that utility is \textbf{strictly} monotonic, 
consumers can increase their consumption of $l$ arbitrarily large such that 
$p a < p \sum_i x_i$. Then, we have a contradiction. 

Now suppose there exists some $l$ such that $p_l = 0$, 
then by the continuity of $B(x)$, there exists 
an allocation $y$ such that $p y < p \sum_i x_i$, where $y$ is defined by
\[
y_k = x_k - \epsilon, \text{ and } y_l = x_l + \delta
\]
with sufficiently small $epsilon$ and sufficiently large $\delta$. 
Again, we obtain a contradiction.

\noindent \textbf{Part 2.}
Finally, we prove that, 
any bundle in $B^i(x_i)$ is unaffordable under $p \in \mathbb{R}^L_{++}$.
Since $p x_i > 0$, there exists a strictly cheaper bundel such that 
\[
p \underline{x}_i < p x_i
\]
It follows that $\alpha y_i + (1 - \alpha) \underline{x}_i < p x_i$ for some $y \in B^i(x_i)$ 
and $\alpha \in (0,1)$.
But by the convexity of $B^i(x_i)$, we have
$\alpha y_i + (1 - \alpha) \underline{x}_i := \tilde{x}_i \in B^i(x_i)$.
Then, summing over individuals, we have
\[
p \tilde{x}_i + \sum_{j \neq i} p x_j < p \sum_{j}  x_j,
\]
which contradicts to the separating hyperplane condition. \(\blacksquare\)


\section{Sufficient conditions for the unique of competitive equilibrium}
